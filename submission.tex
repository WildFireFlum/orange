%%%%%%%%%%%%%%%%%%%%%%%%%%%%%%%%%%%%%%%%%%%%%%%%%%%%%%%%%%%%%%%%%%%%%%%%%%%%%%%%
%2345678901234567890123456789012345678901234567890123456789012345678901234567890
%        1         2         3         4         5         6         7         8

\documentclass[letterpaper, 10 pt, conference]{ieeeconf}  % Comment this line out
                                                          % if you need a4paper
%\documentclass[a4paper, 10pt, conference]{ieeeconf}      % Use this line for a4
                                                          % paper


% The following packages can be found on http:\\www.ctan.org
%\usepackage{graphics} % for pdf, bitmapped graphics files
%\usepackage{epsfig} % for postscript graphics files
%\usepackage{mathptmx} % assumes new font selection scheme installed
%\usepackage{times} % assumes new font selection scheme installed
%\usepackage{amsmath} % assumes amsmath package installed
%\usepackage{amssymb}  % assumes amsmath package installed

\title{\LARGE \bf
KiWi as a Priority Queue
}

%\author{ \parbox{3 in}{\centering Huibert Kwakernaak*
%         \thanks{*Use the $\backslash$thanks command to put information here}\\
%         Faculty of Electrical Engineering, Mathematics and Computer Science\\
%         University of Twente\\
%         7500 AE Enschede, The Netherlands\\
%         {\tt\small h.kwakernaak@autsubmit.com}}
%         \hspace*{ 0.5 in}
%         \parbox{3 in}{ \centering Pradeep Misra**
%         \thanks{**The footnote marks may be inserted manually}\\
%        Department of Electrical Engineering \\
%         Wright State University\\
%         Dayton, OH 45435, USA\\
%         {\tt\small pmisra@cs.wright.edu}}
%}

\author{Ynon Flum$^{1}$ , Tzlil Gonen$^{2}$ and Matan Hasson$^3$ % <-this % stops a space
\thanks{$^1$ynonflum@mail.tau.ac.il}% <-this % stops a space
\thanks{$^2$tzlilgon@mail.tau.ac.il}% <-this % stops a space
\thanks{$^3$matanhasson@mail.tau.ac.il}% <-this % stops a space
}


\begin{document}



\maketitle
\thispagestyle{empty}
\pagestyle{empty}


%%%%%%%%%%%%%%%%%%%%%%%%%%%%%%%%%%%%%%%%%%%%%%%%%%%%%%%%%%%%%%%%%%%%%%%%%%%%%%%%
\begin{abstract}

\textit{KiWi}, an ordered key-value map, can be employed as a large scale priority queue. Evaluation of such implementation is by \textit{Galois} benchmarks.

\end{abstract}


%%%%%%%%%%%%%%%%%%%%%%%%%%%%%%%%%%%%%%%%%%%%%%%%%%%%%%%%%%%%%%%%%%%%%%%%%%%%%%%%
\section{INTRODUCTION}

\textbf{KiWi} \cite{kiwi} An atomic Key-Value-map which efficiently supports simultaneous large scans and real-time access, implements ordered key-value map supporting the following operations:
\begin{itemize}
\item \textit{get(key)}: It is used to return the value for the specified key.
\end{itemize}
\begin{itemize}
\item \textit{put(key,value)}: It is used to insert an entry in this map.
\item \textit{scan(fromKey, toKey)}: It is used to get all the entries in the given interval.
\end{itemize}
Both get() and scan() are wait-free and put() is lock-free. To support wait free scans, KiWi employs multi-versioning. In addition, Data in KiWi is organized as a collection of chunks, which are large blocks of contiguous key ranges. Such data layout is cache-friendly and suitable for non-uniform memory architectures (NUMA).
Rebalance was the main mechanism which was used in the original KiWi to allow the data structure to run without waiting (as it is a lock-free algorithm).
In a rebalance, the threads agree on a list of chunks to change (either remove some, add new ones, or a composition of the two).


\textbf{Galois} \cite{galios} is a framework which allows the execution and benchmarking of many task-based parallel algorithms. The asynchronous execution is based on a priority queue which holds the tasks to be executed. The framework allows its user to control many of the parameters of a concurrent run, we are specifically interested in the ability to choose the scheduling algorithm. The framework contains the OrderedByIntegerMetric (OBIM) implementation of the priority queue. 

In this work we will implement a priority queue based on KiWi and evaluate it using the applications implemented in Galois. In such implementation, the key can be used as a priority and value as the task to be executed.


\section{KiWiPQ}

As the ideas behind the lock-free implementation of KiWi seem interesting enough, we were asked to check whether they could be applied for a priority queue. 
There were, of course some adjustments which had to be made, both to the API, and 
to the data structure itself.
Moreover, the article only describes the algorithm in pseudo-code, and many of the issues which arise when one tries to implement it were not addressed.

The main challenges and differences from the original implementation are as follows:

\subsection{Scans}As scans are not relevant to a priority queue, we removed the associated versioning mechanism which was needed to support them in the original article.

\subsection{Pops}As the original data structure did not support deletions there were out of a rebalance operation, we had to implement a lock free mechanism which allowed us to safely remove elements from the list.

\subsection{Indexing}The original Kiwi indexed its chunks in order to gain performance when searching for a chunk. Yet the authors of the article implemented this mechanism using locks, which in turn damped the performance gain (or outright canceled it).
In our implementation we used a slightly altered implementation of the lock free skiplist to index the chunks.

\subsection{Testing}As we were required to integrate the algorithm into the Galois framework, which is not easily run on a PC due to its intensive hardware requirements, we had to implement the data structure in a modular way in order for us to be able to test it locally,
hence our implementation supports using an allocator of choice, which in the case of the Galois framework was an epoch-based allocator.

\section{RELATED WORK}
As far as we know, no one has previously attempted to convert KiWi's implementation to a priority queue.


\subsection{Evaluation}While \textit{Kiwi} has been evaluated before, it was implemented in Java, and was never used as a priority queue. 
Moreover, \textit{Kiwi} was not implemented in a native language and tested as thoroughly as one might expect, therefore we expect a big performance and performance understanding gain from its integration to the Galois infrastructure.

%%%%%%%%%%%%%%%%%%%%%%%%%%%%%%%%%%%%%%%%%%%%%%%%%%%%%%%%%%%%%%%%%%%%%%%%%%%%%%%%




\begin{thebibliography}{99}

\bibitem{kiwi} Dmitry Basin, Edward Bortnikov, Anastasia Braginsky, Guy GolanGueta, Eshcar Hillel, Idit Keidar, and Moshe Sulamy. 2017. KiWi:
A Key-Value Map for Scalable Real-Time Analytics. In Proceedings
of the 22Nd ACM SIGPLAN Symposium on Principles and Practice of
Parallel Programming (PPoPP ’17). ACM, New York, NY, USA, 357–369.
\bibitem{galios} D. Nguyen, A. Lenharth, and K. Pingali. A lightweight infrastructure
for graph analytics. In Proceedings of ACM Symposium on Operating
Systems Principles, SOSP ’13, pages 456–471, Nov. 2013.







\end{thebibliography}




\end{document}
